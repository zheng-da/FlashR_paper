We present FlashMatrix, a matrix-oriented programming framework for general
data analysis. FlashMatrix scales to large datasets by utilizing commodity SSDs.
It provides a high-level programming interface
for users to write machine learning and data mining algorithms in R and
executes the R implementations in parallel and out of core automatically.
For simplicity and flexibility, the core of FlashMatrix only implements
a small number of generalized matrix operations (GenOps). It reimplements
many matrix operations in R \textit{base} package to provide a familiar
programming environment with GenOps for users. To improve performance,
FlashMatrix uses vectorized user-defined functions (VUDF) to reduce the
overhead of function calls and fuses matrix operations to reduce data movement
between CPU and SSDs.

With appropriate optimizations, we demonstrate that the matrix-oriented functional
programming interface in FlashMatrix can achieve very high performance and
scalability when expressing algorithms in data analysis. We implement multiple
statistics and
machine learning algorithms with R and compare their performance with Spark
MLlib, a high-optimized parallel machine learning library, on large datasets.
The R implementations executed in FlashMatrix significantly outperforms
the implemnetations in Spark MLlib.

As such, FlashMatrix significantly simplifies the programming effort of writing
parallel and out-of-core implementations for large-scale data analysis. It
provides domain experts a familiar programming environment of writing large-scale
implementations. It also significantly increases productivity.

Even though SSDs are still an order of magnitude slower than DRAM, the external-
memory execution of many machine learning and statistics applications in
FlashMatrix can achieve performance comparable to their in-memory execution
with appropriate optimizations. We demonstrate that the I/O throughput of 10 GB/s
is able to saturate CPU for many applications even in a large parallel
machine. As such, the external-memory execution also benefits from many in-memory
optimizations.

FlashMatrix opens a new opportunity for large-scale data analysis.
