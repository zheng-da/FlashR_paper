\section{Experimental evaluation}
We evaluate the performance of FlashMatrix with statistics and machine learning
applications. We implement these applications with the R interface of FlashMatrix
and measure their performance both in memory and on SSDs. We compare their
performance with the implementations in Spark MLlib \cite{}. We further
illustrate the effectiveness of the optimizations deployed in FlashMatrix
both in memory and on SSDs.

We conduct experiments on a non-uniform memory architecture machine with
four Intel Xeon E7-4860 processors, clocked at 2.6 GHz, and 1TB memory of
DDR3-1600. Each processor has 12 cores. The machine has three LSI SAS 9300-8e
host bus adapters (HBA) connected to a SuperMicro storage chassis, in which
24 OCZ Intrepid 3000 SSDs are installed. The 24 SSDs together are capable of
delivering 12 GB/s for read and 10 GB/s for write at maximum. The machine runs
Linux kernel v3.13.0. We use 48 threads for in-memory and semi-external execution
of FlashMatrix as well as Spark.

\subsection{Statistics and Machine learning applications}
We implement multiple important applications in statistics and machine learning.
Spark MLlib provides implementations for these applications. For fair comparison,
we implement the applications with the same algorithms used by Spark MLlib.
\begin{itemize}
	\item Multivariant statistical summary: this computes the minimal value,
		the maximal value, mean, L1 norm, L2 norm, the number of non-zero values
		and variance of each random variable for a given set of observations.
	\item Correlation: this computes Pearson's correlation \cite{} between
		two series of observations and is commonly used in statistics.
	\item Singular value decomposition (SVD) \cite{} factorizes a matrix into
		three matrices: $U$, $\Sigma$ and $V$ such that $A=U \Sigma V^T$, where
		both $U$ and $V$ are orthonormal matrices and $\Sigma$ is a diagonal
		matrix with non-negative diagonals in descending order. SVD is commonly
		used for dimension reduction.
	\item KMeans \cite{kmeans} partitions a set of observations into $k$ clusters
		so that each data point belongs to the cluster with minimal mean. KMeans
		is one of the most popular clustering algorithms and is identified as
		one of the top 10 data mining algorithms \cite{top10}.
	\item Gaussian Mixture Model (GMM) \cite{gmm} is a clustering algorithm that
		assumes observations are sampled under a mixture of Gaussian distributions.
		We implement the expectation maximization (EM) \cite{em} algorithm to fit
		the mixture of Gaussian distributions. This algorithm is also identified
		as one of the top 10 data mining algorithms \cite{top10}.
\end{itemize}

\subsection{Performance of FlashMatrix vs. Spark MLlib}

The performance of FlashMatrix (Figure \ref{perf:fm})

\begin{figure}
	\begin{center}
		\footnotesize
		\begin{tikzpicture}[gnuplot]
%% generated with GNUPLOT 4.6p4 (Lua 5.1; terminal rev. 99, script rev. 100)
%% Fri 08 Apr 2016 06:45:40 PM EDT
\path (0.000,0.000) rectangle (8.382,4.572);
\gpcolor{color=gp lt color border}
\gpsetlinetype{gp lt border}
\gpsetlinewidth{1.00}
\draw[gp path] (1.688,0.616)--(1.868,0.616);
\draw[gp path] (7.829,0.616)--(7.649,0.616);
\node[gp node right] at (1.504,0.616) { 0.1};
\draw[gp path] (1.688,0.830)--(1.778,0.830);
\draw[gp path] (7.829,0.830)--(7.739,0.830);
\draw[gp path] (1.688,1.113)--(1.778,1.113);
\draw[gp path] (7.829,1.113)--(7.739,1.113);
\draw[gp path] (1.688,1.258)--(1.778,1.258);
\draw[gp path] (7.829,1.258)--(7.739,1.258);
\draw[gp path] (1.688,1.327)--(1.868,1.327);
\draw[gp path] (7.829,1.327)--(7.649,1.327);
\node[gp node right] at (1.504,1.327) { 1};
\draw[gp path] (1.688,1.542)--(1.778,1.542);
\draw[gp path] (7.829,1.542)--(7.739,1.542);
\draw[gp path] (1.688,1.825)--(1.778,1.825);
\draw[gp path] (7.829,1.825)--(7.739,1.825);
\draw[gp path] (1.688,1.970)--(1.778,1.970);
\draw[gp path] (7.829,1.970)--(7.739,1.970);
\draw[gp path] (1.688,2.039)--(1.868,2.039);
\draw[gp path] (7.829,2.039)--(7.649,2.039);
\node[gp node right] at (1.504,2.039) { 10};
\draw[gp path] (1.688,2.253)--(1.778,2.253);
\draw[gp path] (7.829,2.253)--(7.739,2.253);
\draw[gp path] (1.688,2.536)--(1.778,2.536);
\draw[gp path] (7.829,2.536)--(7.739,2.536);
\draw[gp path] (1.688,2.681)--(1.778,2.681);
\draw[gp path] (7.829,2.681)--(7.739,2.681);
\draw[gp path] (1.688,2.750)--(1.868,2.750);
\draw[gp path] (7.829,2.750)--(7.649,2.750);
\node[gp node right] at (1.504,2.750) { 100};
\draw[gp path] (1.688,2.964)--(1.778,2.964);
\draw[gp path] (7.829,2.964)--(7.739,2.964);
\draw[gp path] (1.688,3.248)--(1.778,3.248);
\draw[gp path] (7.829,3.248)--(7.739,3.248);
\draw[gp path] (1.688,3.393)--(1.778,3.393);
\draw[gp path] (7.829,3.393)--(7.739,3.393);
\draw[gp path] (1.688,3.462)--(1.868,3.462);
\draw[gp path] (7.829,3.462)--(7.649,3.462);
\node[gp node right] at (1.504,3.462) { 1000};
\draw[gp path] (2.712,0.616)--(2.712,0.796);
\draw[gp path] (2.712,3.587)--(2.712,3.407);
\node[gp node center] at (2.712,0.308) {summary};
\draw[gp path] (3.735,0.616)--(3.735,0.796);
\draw[gp path] (3.735,3.587)--(3.735,3.407);
\node[gp node center] at (3.735,0.308) {cor};
\draw[gp path] (4.759,0.616)--(4.759,0.796);
\draw[gp path] (4.759,3.587)--(4.759,3.407);
\node[gp node center] at (4.759,0.308) {SVD};
\draw[gp path] (5.782,0.616)--(5.782,0.796);
\draw[gp path] (5.782,3.587)--(5.782,3.407);
\node[gp node center] at (5.782,0.308) {KMeans};
\draw[gp path] (6.806,0.616)--(6.806,0.796);
\draw[gp path] (6.806,3.587)--(6.806,3.407);
\node[gp node center] at (6.806,0.308) {GMM};
\draw[gp path] (1.688,3.587)--(1.688,0.616)--(7.829,0.616)--(7.829,3.587)--cycle;
\node[gp node center,rotate=-270] at (0.246,2.101) {Runtime (s)};
\node[gp node right] at (2.372,4.238) {FM-IM};
\def\gpfillpath{(2.556,4.161)--(3.472,4.161)--(3.472,4.315)--(2.556,4.315)--cycle}
\gpfill{color=gpbgfillcolor} \gpfillpath;
\gpfill{color=gp lt color 0,gp pattern 0,pattern color=.} \gpfillpath;
\gpcolor{color=gp lt color 0}
\gpsetlinetype{gp lt plot 0}
\draw[gp path] (2.556,4.161)--(3.472,4.161)--(3.472,4.315)--(2.556,4.315)--cycle;
\def\gpfillpath{(2.507,0.616)--(2.713,0.616)--(2.713,1.955)--(2.507,1.955)--cycle}
\gpfill{color=gpbgfillcolor} \gpfillpath;
\gpfill{color=gp lt color 0,gp pattern 0,pattern color=.} \gpfillpath;
\draw[gp path] (2.507,0.616)--(2.507,1.954)--(2.712,1.954)--(2.712,0.616)--cycle;
\def\gpfillpath{(3.530,0.616)--(3.736,0.616)--(3.736,2.185)--(3.530,2.185)--cycle}
\gpfill{color=gpbgfillcolor} \gpfillpath;
\gpfill{color=gp lt color 0,gp pattern 0,pattern color=.} \gpfillpath;
\draw[gp path] (3.530,0.616)--(3.530,2.184)--(3.735,2.184)--(3.735,0.616)--cycle;
\def\gpfillpath{(4.554,0.616)--(4.760,0.616)--(4.760,2.125)--(4.554,2.125)--cycle}
\gpfill{color=gpbgfillcolor} \gpfillpath;
\gpfill{color=gp lt color 0,gp pattern 0,pattern color=.} \gpfillpath;
\draw[gp path] (4.554,0.616)--(4.554,2.124)--(4.759,2.124)--(4.759,0.616)--cycle;
\def\gpfillpath{(5.577,0.616)--(5.783,0.616)--(5.783,2.222)--(5.577,2.222)--cycle}
\gpfill{color=gpbgfillcolor} \gpfillpath;
\gpfill{color=gp lt color 0,gp pattern 0,pattern color=.} \gpfillpath;
\draw[gp path] (5.577,0.616)--(5.577,2.221)--(5.782,2.221)--(5.782,0.616)--cycle;
\def\gpfillpath{(6.601,0.616)--(6.807,0.616)--(6.807,3.191)--(6.601,3.191)--cycle}
\gpfill{color=gpbgfillcolor} \gpfillpath;
\gpfill{color=gp lt color 0,gp pattern 0,pattern color=.} \gpfillpath;
\draw[gp path] (6.601,0.616)--(6.601,3.190)--(6.806,3.190)--(6.806,0.616)--cycle;
\gpcolor{color=gp lt color border}
\node[gp node right] at (4.576,4.238) {FM-EM};
\def\gpfillpath{(4.760,4.161)--(5.676,4.161)--(5.676,4.315)--(4.760,4.315)--cycle}
\gpfill{color=gpbgfillcolor} \gpfillpath;
\gpfill{color=gp lt color 1,gp pattern 1,pattern color=.} \gpfillpath;
\gpcolor{color=gp lt color 1}
\gpsetlinetype{gp lt plot 1}
\draw[gp path] (4.760,4.161)--(5.676,4.161)--(5.676,4.315)--(4.760,4.315)--cycle;
\def\gpfillpath{(2.712,0.616)--(2.917,0.616)--(2.917,2.298)--(2.712,2.298)--cycle}
\gpfill{color=gpbgfillcolor} \gpfillpath;
\gpfill{color=gp lt color 1,gp pattern 1,pattern color=.} \gpfillpath;
\draw[gp path] (2.712,0.616)--(2.712,2.297)--(2.916,2.297)--(2.916,0.616)--cycle;
\def\gpfillpath{(3.735,0.616)--(3.941,0.616)--(3.941,2.298)--(3.735,2.298)--cycle}
\gpfill{color=gpbgfillcolor} \gpfillpath;
\gpfill{color=gp lt color 1,gp pattern 1,pattern color=.} \gpfillpath;
\draw[gp path] (3.735,0.616)--(3.735,2.297)--(3.940,2.297)--(3.940,0.616)--cycle;
\def\gpfillpath{(4.759,0.616)--(4.964,0.616)--(4.964,2.294)--(4.759,2.294)--cycle}
\gpfill{color=gpbgfillcolor} \gpfillpath;
\gpfill{color=gp lt color 1,gp pattern 1,pattern color=.} \gpfillpath;
\draw[gp path] (4.759,0.616)--(4.759,2.293)--(4.963,2.293)--(4.963,0.616)--cycle;
\def\gpfillpath{(5.782,0.616)--(5.988,0.616)--(5.988,2.302)--(5.782,2.302)--cycle}
\gpfill{color=gpbgfillcolor} \gpfillpath;
\gpfill{color=gp lt color 1,gp pattern 1,pattern color=.} \gpfillpath;
\draw[gp path] (5.782,0.616)--(5.782,2.301)--(5.987,2.301)--(5.987,0.616)--cycle;
\def\gpfillpath{(6.806,0.616)--(7.011,0.616)--(7.011,3.216)--(6.806,3.216)--cycle}
\gpfill{color=gpbgfillcolor} \gpfillpath;
\gpfill{color=gp lt color 1,gp pattern 1,pattern color=.} \gpfillpath;
\draw[gp path] (6.806,0.616)--(6.806,3.215)--(7.010,3.215)--(7.010,0.616)--cycle;
\gpcolor{color=gp lt color border}
\node[gp node right] at (6.780,4.238) {MLlib};
\def\gpfillpath{(6.964,4.161)--(7.880,4.161)--(7.880,4.315)--(6.964,4.315)--cycle}
\gpfill{color=gpbgfillcolor} \gpfillpath;
\gpfill{color=gp lt color 2,gp pattern 2,pattern color=.} \gpfillpath;
\gpcolor{color=gp lt color 2}
\gpsetlinetype{gp lt plot 2}
\draw[gp path] (6.964,4.161)--(7.880,4.161)--(7.880,4.315)--(6.964,4.315)--cycle;
\def\gpfillpath{(2.916,0.616)--(3.122,0.616)--(3.122,2.701)--(2.916,2.701)--cycle}
\gpfill{color=gpbgfillcolor} \gpfillpath;
\gpfill{color=gp lt color 2,gp pattern 2,pattern color=.} \gpfillpath;
\draw[gp path] (2.916,0.616)--(2.916,2.700)--(3.121,2.700)--(3.121,0.616)--cycle;
\def\gpfillpath{(3.940,0.616)--(4.145,0.616)--(4.145,3.006)--(3.940,3.006)--cycle}
\gpfill{color=gpbgfillcolor} \gpfillpath;
\gpfill{color=gp lt color 2,gp pattern 2,pattern color=.} \gpfillpath;
\draw[gp path] (3.940,0.616)--(3.940,3.005)--(4.144,3.005)--(4.144,0.616)--cycle;
\def\gpfillpath{(4.963,0.616)--(5.169,0.616)--(5.169,2.739)--(4.963,2.739)--cycle}
\gpfill{color=gpbgfillcolor} \gpfillpath;
\gpfill{color=gp lt color 2,gp pattern 2,pattern color=.} \gpfillpath;
\draw[gp path] (4.963,0.616)--(4.963,2.738)--(5.168,2.738)--(5.168,0.616)--cycle;
\def\gpfillpath{(5.987,0.616)--(6.192,0.616)--(6.192,2.875)--(5.987,2.875)--cycle}
\gpfill{color=gpbgfillcolor} \gpfillpath;
\gpfill{color=gp lt color 2,gp pattern 2,pattern color=.} \gpfillpath;
\draw[gp path] (5.987,0.616)--(5.987,2.874)--(6.191,2.874)--(6.191,0.616)--cycle;
\def\gpfillpath{(7.010,0.616)--(7.216,0.616)--(7.216,3.576)--(7.010,3.576)--cycle}
\gpfill{color=gpbgfillcolor} \gpfillpath;
\gpfill{color=gp lt color 2,gp pattern 2,pattern color=.} \gpfillpath;
\draw[gp path] (7.010,0.616)--(7.010,3.575)--(7.215,3.575)--(7.215,0.616)--cycle;
\gpcolor{color=gp lt color border}
\gpsetlinetype{gp lt border}
\draw[gp path] (1.688,3.587)--(1.688,0.616)--(7.829,0.616)--(7.829,3.587)--cycle;
%% coordinates of the plot area
\gpdefrectangularnode{gp plot 1}{\pgfpoint{1.688cm}{0.616cm}}{\pgfpoint{7.829cm}{3.587cm}}
\end{tikzpicture}
%% gnuplot variables

		\caption{The performance of FlashMatrix both in memory and on SSDs
		compared with Spark.}
		\label{perf:fm}
	\end{center}
\end{figure}

\subsection{Performance of FlashMatrix vs. R}

\subsection{Performance of FlashMatrix vs. C/C++}

\subsection{Microbenchmarks}

The impact of operation fuse.
the benefit of I/O-level partition.
the benefit of CPU-level partition.

Overhead of function calls.
