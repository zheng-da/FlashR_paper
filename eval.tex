\section{Experimental evaluation}
We evaluate the performance of FlashMatrix with statistics and machine learning
applications. We implement these applications with the R interface of FlashMatrix
and measure their performance both in memory and on SSDs. We compare their
performance with the implementations in Spark MLlib \cite{}. We further
illustrate the effectiveness of the optimizations deployed in FlashMatrix
both in memory and on SSDs.

We conduct experiments on a non-uniform memory architecture machine with
four Intel Xeon E7-4860 processors, clocked at 2.6 GHz, and 1TB memory of
DDR3-1600. Each processor has 12 cores. The machine has three LSI SAS 9300-8e
host bus adapters (HBA) connected to a SuperMicro storage chassis, in which
24 OCZ Intrepid 3000 SSDs are installed. The 24 SSDs together are capable of
delivering 12 GB/s for read and 10 GB/s for write at maximum. The machine runs
Linux kernel v3.13.0. We use 48 threads for in-memory and semi-external execution
of FlashMatrix as well as Spark.

\subsection{Statistics and Machine learning applications} \label{sec:apps}
We implement multiple important applications in the field of statistics and
machine learning in FlashMatrix. These applications are implemented completely
with R and rely on FlashMatrix to perform computation in parallel and out of
core.
\begin{itemize}
	\item Multivariant statistical summary: this computes the minimal value,
		the maximal value, mean, L1 norm, L2 norm, the number of non-zero values
		and variance of each random variable for a given set of observations.
	\item Correlation: this computes Pearson's correlation \cite{} between
		two series of observations and is commonly used in statistics.
	\item Singular value decomposition (SVD) \cite{} factorizes a matrix into
		three matrices: $U$, $\Sigma$ and $V$ such that $A=U \Sigma V^T$, where
		both $U$ and $V$ are orthonormal matrices and $\Sigma$ is a diagonal
		matrix with non-negative diagonals in descending order. SVD is commonly
		used for dimension reduction.
	\item KMeans \cite{kmeans} partitions a set of observations into $k$ clusters
		so that each data point belongs to the cluster with minimal mean. KMeans
		is one of the most popular clustering algorithms and is identified as
		one of the top 10 data mining algorithms \cite{top10}.
	\item Gaussian Mixture Model (GMM) \cite{gmm} is a clustering algorithm that
		assumes observations are sampled under a mixture of Gaussian distributions.
		We implement the expectation maximization (EM) \cite{em} algorithm to fit
		the mixture of Gaussian distributions. This algorithm is also identified
		as one of the top 10 data mining algorithms \cite{top10}.
\end{itemize}

\dz{We should describe what datasets the algorithms usually take}

\subsection{Performance of FlashMatrix in memory and on SSDs}

The computation complexity of statistics computation and machine learning
algorithms may vary for different datasets and different settings. To illusrate
the performance of the applications thoroughly, we measure the performance
of the first three applications on datasets with 10 million data points each
with various numbers of features from 8 to 512. We measure the performance of
the clustering algorithms (KMeans and GMM) on a dataset with 65 million data
points each with 32 features and vary the number of clusters from 2 to 64.

\begin{figure}
	\begin{center}
		\footnotesize
		\begin{tikzpicture}[gnuplot]
%% generated with GNUPLOT 4.6p4 (Lua 5.1; terminal rev. 99, script rev. 100)
%% Sun 10 Apr 2016 09:18:38 PM EDT
\path (0.000,0.000) rectangle (8.382,4.572);
\gpcolor{color=gp lt color border}
\gpsetlinetype{gp lt border}
\gpsetlinewidth{1.00}
\draw[gp path] (1.196,0.985)--(1.376,0.985);
\draw[gp path] (7.829,0.985)--(7.649,0.985);
\node[gp node right] at (1.012,0.985) { 0};
\draw[gp path] (1.196,1.505)--(1.376,1.505);
\draw[gp path] (7.829,1.505)--(7.649,1.505);
\node[gp node right] at (1.012,1.505) { 0.2};
\draw[gp path] (1.196,2.026)--(1.376,2.026);
\draw[gp path] (7.829,2.026)--(7.649,2.026);
\node[gp node right] at (1.012,2.026) { 0.4};
\draw[gp path] (1.196,2.546)--(1.376,2.546);
\draw[gp path] (7.829,2.546)--(7.649,2.546);
\node[gp node right] at (1.012,2.546) { 0.6};
\draw[gp path] (1.196,3.067)--(1.376,3.067);
\draw[gp path] (7.829,3.067)--(7.649,3.067);
\node[gp node right] at (1.012,3.067) { 0.8};
\draw[gp path] (1.196,3.587)--(1.376,3.587);
\draw[gp path] (7.829,3.587)--(7.649,3.587);
\node[gp node right] at (1.012,3.587) { 1};
\draw[gp path] (1.196,0.985)--(1.196,1.165);
\draw[gp path] (1.196,3.587)--(1.196,3.407);
\node[gp node center] at (1.196,0.677) {8};
\draw[gp path] (3.407,0.985)--(3.407,1.165);
\draw[gp path] (3.407,3.587)--(3.407,3.407);
\node[gp node center] at (3.407,0.677) {32};
\draw[gp path] (5.618,0.985)--(5.618,1.165);
\draw[gp path] (5.618,3.587)--(5.618,3.407);
\node[gp node center] at (5.618,0.677) {128};
\draw[gp path] (7.829,0.985)--(7.829,1.165);
\draw[gp path] (7.829,3.587)--(7.829,3.407);
\node[gp node center] at (7.829,0.677) {512};
\draw[gp path] (1.196,3.587)--(1.196,0.985)--(7.829,0.985)--(7.829,3.587)--cycle;
\node[gp node center] at (4.512,0.215) {The number of columns};
\node[gp node right] at (1.942,4.238) {summary};
\gpcolor{color=gp lt color 0}
\gpsetlinetype{gp lt plot 0}
\draw[gp path] (2.126,4.238)--(3.042,4.238);
\draw[gp path] (1.196,1.662)--(3.407,1.844)--(5.618,2.364)--(7.829,2.598);
\gpsetpointsize{4.00}
\gppoint{gp mark 1}{(1.196,1.662)}
\gppoint{gp mark 1}{(3.407,1.844)}
\gppoint{gp mark 1}{(5.618,2.364)}
\gppoint{gp mark 1}{(7.829,2.598)}
\gppoint{gp mark 1}{(2.584,4.238)}
\gpcolor{color=gp lt color border}
\node[gp node right] at (4.514,4.238) {cor};
\gpcolor{color=gp lt color 1}
\gpsetlinetype{gp lt plot 1}
\draw[gp path] (4.698,4.238)--(5.614,4.238);
\draw[gp path] (1.196,1.766)--(3.407,2.520)--(5.618,2.989)--(7.829,2.989);
\gppoint{gp mark 2}{(1.196,1.766)}
\gppoint{gp mark 2}{(3.407,2.520)}
\gppoint{gp mark 2}{(5.618,2.989)}
\gppoint{gp mark 2}{(7.829,2.989)}
\gppoint{gp mark 2}{(5.156,4.238)}
\gpcolor{color=gp lt color border}
\node[gp node right] at (7.086,4.238) {SVD};
\gpcolor{color=gp lt color 2}
\gpsetlinetype{gp lt plot 2}
\draw[gp path] (7.270,4.238)--(8.186,4.238);
\draw[gp path] (1.196,1.531)--(3.407,2.390)--(5.618,3.561)--(7.829,3.535);
\gppoint{gp mark 3}{(1.196,1.531)}
\gppoint{gp mark 3}{(3.407,2.390)}
\gppoint{gp mark 3}{(5.618,3.561)}
\gppoint{gp mark 3}{(7.829,3.535)}
\gppoint{gp mark 3}{(7.728,4.238)}
\gpcolor{color=gp lt color border}
\gpsetlinetype{gp lt border}
\draw[gp path] (1.196,3.587)--(1.196,0.985)--(7.829,0.985)--(7.829,3.587)--cycle;
%% coordinates of the plot area
\gpdefrectangularnode{gp plot 1}{\pgfpoint{1.196cm}{0.985cm}}{\pgfpoint{7.829cm}{3.587cm}}
\end{tikzpicture}
%% gnuplot variables

		\caption{The relative external-memory performance of FlashMatrix for
			statistics computation on a dataset with the number of features
		varying from 8 to 512, normalized by its in-memory performance.}
		\label{perf:stat}
	\end{center}
\end{figure}

As the number of features in the datasets or the number of clusters increases,
the performance gap between in-memory and external-memory execution
narrows and eventually the external-memory performance gets almost 100\%
of in-memory performance (Figure \ref{perf:stat} and \ref{perf:clust}).
The main computation overhead in correlation and SVD is cross product on
the $n \times p$ input matrix, which has the computation complexity of
$O(n \times p^2)$ and I/O complexity of $O(n \times p)$. As such, when
the number of features in the dataset gets larger, the ratio of computation
and I/O in correlation and SVD gets larger and eventually CPU becomes
the bottleneck. Similarly, the computation of KMeans and GMM increases
more rapidly than I/O as the number of clusters gets larger, and get
dominated by their computation even when the number of clusters is not
very large.

\begin{figure}
	\begin{center}
		\footnotesize
		\begin{tikzpicture}[gnuplot]
%% generated with GNUPLOT 4.6p4 (Lua 5.1; terminal rev. 99, script rev. 100)
%% Wed 30 Mar 2016 08:12:03 AM EDT
\path (0.000,0.000) rectangle (8.382,4.572);
\gpcolor{color=gp lt color border}
\gpsetlinetype{gp lt border}
\gpsetlinewidth{1.00}
\draw[gp path] (1.196,0.985)--(1.376,0.985);
\draw[gp path] (7.829,0.985)--(7.649,0.985);
\node[gp node right] at (1.012,0.985) { 0};
\draw[gp path] (1.196,1.445)--(1.376,1.445);
\draw[gp path] (7.829,1.445)--(7.649,1.445);
\node[gp node right] at (1.012,1.445) { 0.5};
\draw[gp path] (1.196,1.904)--(1.376,1.904);
\draw[gp path] (7.829,1.904)--(7.649,1.904);
\node[gp node right] at (1.012,1.904) { 1};
\draw[gp path] (1.196,2.364)--(1.376,2.364);
\draw[gp path] (7.829,2.364)--(7.649,2.364);
\node[gp node right] at (1.012,2.364) { 1.5};
\draw[gp path] (1.196,2.824)--(1.376,2.824);
\draw[gp path] (7.829,2.824)--(7.649,2.824);
\node[gp node right] at (1.012,2.824) { 2};
\draw[gp path] (1.196,3.284)--(1.376,3.284);
\draw[gp path] (7.829,3.284)--(7.649,3.284);
\node[gp node right] at (1.012,3.284) { 2.5};
\draw[gp path] (1.196,3.743)--(1.376,3.743);
\draw[gp path] (7.829,3.743)--(7.649,3.743);
\node[gp node right] at (1.012,3.743) { 3};
\draw[gp path] (1.196,4.203)--(1.376,4.203);
\draw[gp path] (7.829,4.203)--(7.649,4.203);
\node[gp node right] at (1.012,4.203) { 3.5};
\draw[gp path] (1.196,0.985)--(1.196,1.165);
\draw[gp path] (1.196,4.203)--(1.196,4.023);
\node[gp node center] at (1.196,0.677) {2};
\draw[gp path] (2.523,0.985)--(2.523,1.165);
\draw[gp path] (2.523,4.203)--(2.523,4.023);
\node[gp node center] at (2.523,0.677) {4};
\draw[gp path] (3.849,0.985)--(3.849,1.165);
\draw[gp path] (3.849,4.203)--(3.849,4.023);
\node[gp node center] at (3.849,0.677) {8};
\draw[gp path] (5.176,0.985)--(5.176,1.165);
\draw[gp path] (5.176,4.203)--(5.176,4.023);
\node[gp node center] at (5.176,0.677) {16};
\draw[gp path] (6.502,0.985)--(6.502,1.165);
\draw[gp path] (6.502,4.203)--(6.502,4.023);
\node[gp node center] at (6.502,0.677) {32};
\draw[gp path] (7.829,0.985)--(7.829,1.165);
\draw[gp path] (7.829,4.203)--(7.829,4.023);
\node[gp node center] at (7.829,0.677) {64};
\draw[gp path] (1.196,4.203)--(1.196,0.985)--(7.829,0.985)--(7.829,4.203)--cycle;
\node[gp node center] at (4.512,0.215) {The number of columns};
\node[gp node right] at (6.361,3.869) {KMeans};
\gpcolor{color=gp lt color 0}
\gpsetlinetype{gp lt plot 0}
\draw[gp path] (6.545,3.869)--(7.461,3.869);
\draw[gp path] (1.196,3.964)--(2.523,3.247)--(3.849,2.566)--(5.176,2.153)--(6.502,2.061)%
  --(7.829,1.914);
\gpsetpointsize{4.00}
\gppoint{gp mark 1}{(1.196,3.964)}
\gppoint{gp mark 1}{(2.523,3.247)}
\gppoint{gp mark 1}{(3.849,2.566)}
\gppoint{gp mark 1}{(5.176,2.153)}
\gppoint{gp mark 1}{(6.502,2.061)}
\gppoint{gp mark 1}{(7.829,1.914)}
\gppoint{gp mark 1}{(7.003,3.869)}
\gpcolor{color=gp lt color border}
\node[gp node right] at (6.361,3.561) {GMM};
\gpcolor{color=gp lt color 1}
\gpsetlinetype{gp lt plot 1}
\draw[gp path] (6.545,3.561)--(7.461,3.561);
\draw[gp path] (1.196,2.953)--(2.523,2.309)--(3.849,2.125)--(5.176,2.116)--(6.502,2.015)%
  --(7.829,1.996);
\gppoint{gp mark 2}{(1.196,2.953)}
\gppoint{gp mark 2}{(2.523,2.309)}
\gppoint{gp mark 2}{(3.849,2.125)}
\gppoint{gp mark 2}{(5.176,2.116)}
\gppoint{gp mark 2}{(6.502,2.015)}
\gppoint{gp mark 2}{(7.829,1.996)}
\gppoint{gp mark 2}{(7.003,3.561)}
\gpcolor{color=gp lt color border}
\gpsetlinetype{gp lt border}
\draw[gp path] (1.196,4.203)--(1.196,0.985)--(7.829,0.985)--(7.829,4.203)--cycle;
%% coordinates of the plot area
\gpdefrectangularnode{gp plot 1}{\pgfpoint{1.196cm}{0.985cm}}{\pgfpoint{7.829cm}{4.203cm}}
\end{tikzpicture}
%% gnuplot variables

		\caption{The relative external-memory performance of FlashMatrix for
			clustering algorithms with different numbers of clusters, normalized
		by its in-memory performance.}
		\label{perf:clust}
	\end{center}
\end{figure}

\subsection{Performance of FlashMatrix vs. Spark MLlib}

We compare the performance of the FlashMatrix implementations of the applications
in Section \ref{sec:apps} with the ones in Spark MLlib. Spark MLlib provides
implementations for these applications. We implement the applications with
the same algorithms used by Spark MLlib. For fair comparison, we runs the Spark
implementations with their native Scala interface and use a very large heap size
to ensure that all input data is cached in memory. We measure the performance
of both in-memory and out-of-core execution of the applications in FlashMatrix.

The applications in FlashMatrix significantly outperform Spark MLlib (Figure
\ref{perf:fm}). When the applications require more computation, the out-of-core
execution of FlashMatrix significantly outperforms Spark MLlib. Multivariant
statistical summary is the only application that does not outperform Spark
when executed out of core because its out-of-core execution is bottlenecked
by I/O. For applications such as correlation and GMM, even though both FlashMatrix
and Spark implementations heavily rely on BLAS for matrix multiplication,
FlashMatrix can still significantly outperform Spark thanks to matrix operation
fusion and two-level partitioning for computation in FlashMatrix.

\begin{figure}
	\begin{center}
		\footnotesize
		\begin{tikzpicture}[gnuplot]
%% generated with GNUPLOT 4.6p4 (Lua 5.1; terminal rev. 99, script rev. 100)
%% Fri 08 Apr 2016 06:45:40 PM EDT
\path (0.000,0.000) rectangle (8.382,4.572);
\gpcolor{color=gp lt color border}
\gpsetlinetype{gp lt border}
\gpsetlinewidth{1.00}
\draw[gp path] (1.688,0.616)--(1.868,0.616);
\draw[gp path] (7.829,0.616)--(7.649,0.616);
\node[gp node right] at (1.504,0.616) { 0.1};
\draw[gp path] (1.688,0.830)--(1.778,0.830);
\draw[gp path] (7.829,0.830)--(7.739,0.830);
\draw[gp path] (1.688,1.113)--(1.778,1.113);
\draw[gp path] (7.829,1.113)--(7.739,1.113);
\draw[gp path] (1.688,1.258)--(1.778,1.258);
\draw[gp path] (7.829,1.258)--(7.739,1.258);
\draw[gp path] (1.688,1.327)--(1.868,1.327);
\draw[gp path] (7.829,1.327)--(7.649,1.327);
\node[gp node right] at (1.504,1.327) { 1};
\draw[gp path] (1.688,1.542)--(1.778,1.542);
\draw[gp path] (7.829,1.542)--(7.739,1.542);
\draw[gp path] (1.688,1.825)--(1.778,1.825);
\draw[gp path] (7.829,1.825)--(7.739,1.825);
\draw[gp path] (1.688,1.970)--(1.778,1.970);
\draw[gp path] (7.829,1.970)--(7.739,1.970);
\draw[gp path] (1.688,2.039)--(1.868,2.039);
\draw[gp path] (7.829,2.039)--(7.649,2.039);
\node[gp node right] at (1.504,2.039) { 10};
\draw[gp path] (1.688,2.253)--(1.778,2.253);
\draw[gp path] (7.829,2.253)--(7.739,2.253);
\draw[gp path] (1.688,2.536)--(1.778,2.536);
\draw[gp path] (7.829,2.536)--(7.739,2.536);
\draw[gp path] (1.688,2.681)--(1.778,2.681);
\draw[gp path] (7.829,2.681)--(7.739,2.681);
\draw[gp path] (1.688,2.750)--(1.868,2.750);
\draw[gp path] (7.829,2.750)--(7.649,2.750);
\node[gp node right] at (1.504,2.750) { 100};
\draw[gp path] (1.688,2.964)--(1.778,2.964);
\draw[gp path] (7.829,2.964)--(7.739,2.964);
\draw[gp path] (1.688,3.248)--(1.778,3.248);
\draw[gp path] (7.829,3.248)--(7.739,3.248);
\draw[gp path] (1.688,3.393)--(1.778,3.393);
\draw[gp path] (7.829,3.393)--(7.739,3.393);
\draw[gp path] (1.688,3.462)--(1.868,3.462);
\draw[gp path] (7.829,3.462)--(7.649,3.462);
\node[gp node right] at (1.504,3.462) { 1000};
\draw[gp path] (2.712,0.616)--(2.712,0.796);
\draw[gp path] (2.712,3.587)--(2.712,3.407);
\node[gp node center] at (2.712,0.308) {summary};
\draw[gp path] (3.735,0.616)--(3.735,0.796);
\draw[gp path] (3.735,3.587)--(3.735,3.407);
\node[gp node center] at (3.735,0.308) {cor};
\draw[gp path] (4.759,0.616)--(4.759,0.796);
\draw[gp path] (4.759,3.587)--(4.759,3.407);
\node[gp node center] at (4.759,0.308) {SVD};
\draw[gp path] (5.782,0.616)--(5.782,0.796);
\draw[gp path] (5.782,3.587)--(5.782,3.407);
\node[gp node center] at (5.782,0.308) {KMeans};
\draw[gp path] (6.806,0.616)--(6.806,0.796);
\draw[gp path] (6.806,3.587)--(6.806,3.407);
\node[gp node center] at (6.806,0.308) {GMM};
\draw[gp path] (1.688,3.587)--(1.688,0.616)--(7.829,0.616)--(7.829,3.587)--cycle;
\node[gp node center,rotate=-270] at (0.246,2.101) {Runtime (s)};
\node[gp node right] at (2.372,4.238) {FM-IM};
\def\gpfillpath{(2.556,4.161)--(3.472,4.161)--(3.472,4.315)--(2.556,4.315)--cycle}
\gpfill{color=gpbgfillcolor} \gpfillpath;
\gpfill{color=gp lt color 0,gp pattern 0,pattern color=.} \gpfillpath;
\gpcolor{color=gp lt color 0}
\gpsetlinetype{gp lt plot 0}
\draw[gp path] (2.556,4.161)--(3.472,4.161)--(3.472,4.315)--(2.556,4.315)--cycle;
\def\gpfillpath{(2.507,0.616)--(2.713,0.616)--(2.713,1.955)--(2.507,1.955)--cycle}
\gpfill{color=gpbgfillcolor} \gpfillpath;
\gpfill{color=gp lt color 0,gp pattern 0,pattern color=.} \gpfillpath;
\draw[gp path] (2.507,0.616)--(2.507,1.954)--(2.712,1.954)--(2.712,0.616)--cycle;
\def\gpfillpath{(3.530,0.616)--(3.736,0.616)--(3.736,2.185)--(3.530,2.185)--cycle}
\gpfill{color=gpbgfillcolor} \gpfillpath;
\gpfill{color=gp lt color 0,gp pattern 0,pattern color=.} \gpfillpath;
\draw[gp path] (3.530,0.616)--(3.530,2.184)--(3.735,2.184)--(3.735,0.616)--cycle;
\def\gpfillpath{(4.554,0.616)--(4.760,0.616)--(4.760,2.125)--(4.554,2.125)--cycle}
\gpfill{color=gpbgfillcolor} \gpfillpath;
\gpfill{color=gp lt color 0,gp pattern 0,pattern color=.} \gpfillpath;
\draw[gp path] (4.554,0.616)--(4.554,2.124)--(4.759,2.124)--(4.759,0.616)--cycle;
\def\gpfillpath{(5.577,0.616)--(5.783,0.616)--(5.783,2.222)--(5.577,2.222)--cycle}
\gpfill{color=gpbgfillcolor} \gpfillpath;
\gpfill{color=gp lt color 0,gp pattern 0,pattern color=.} \gpfillpath;
\draw[gp path] (5.577,0.616)--(5.577,2.221)--(5.782,2.221)--(5.782,0.616)--cycle;
\def\gpfillpath{(6.601,0.616)--(6.807,0.616)--(6.807,3.191)--(6.601,3.191)--cycle}
\gpfill{color=gpbgfillcolor} \gpfillpath;
\gpfill{color=gp lt color 0,gp pattern 0,pattern color=.} \gpfillpath;
\draw[gp path] (6.601,0.616)--(6.601,3.190)--(6.806,3.190)--(6.806,0.616)--cycle;
\gpcolor{color=gp lt color border}
\node[gp node right] at (4.576,4.238) {FM-EM};
\def\gpfillpath{(4.760,4.161)--(5.676,4.161)--(5.676,4.315)--(4.760,4.315)--cycle}
\gpfill{color=gpbgfillcolor} \gpfillpath;
\gpfill{color=gp lt color 1,gp pattern 1,pattern color=.} \gpfillpath;
\gpcolor{color=gp lt color 1}
\gpsetlinetype{gp lt plot 1}
\draw[gp path] (4.760,4.161)--(5.676,4.161)--(5.676,4.315)--(4.760,4.315)--cycle;
\def\gpfillpath{(2.712,0.616)--(2.917,0.616)--(2.917,2.298)--(2.712,2.298)--cycle}
\gpfill{color=gpbgfillcolor} \gpfillpath;
\gpfill{color=gp lt color 1,gp pattern 1,pattern color=.} \gpfillpath;
\draw[gp path] (2.712,0.616)--(2.712,2.297)--(2.916,2.297)--(2.916,0.616)--cycle;
\def\gpfillpath{(3.735,0.616)--(3.941,0.616)--(3.941,2.298)--(3.735,2.298)--cycle}
\gpfill{color=gpbgfillcolor} \gpfillpath;
\gpfill{color=gp lt color 1,gp pattern 1,pattern color=.} \gpfillpath;
\draw[gp path] (3.735,0.616)--(3.735,2.297)--(3.940,2.297)--(3.940,0.616)--cycle;
\def\gpfillpath{(4.759,0.616)--(4.964,0.616)--(4.964,2.294)--(4.759,2.294)--cycle}
\gpfill{color=gpbgfillcolor} \gpfillpath;
\gpfill{color=gp lt color 1,gp pattern 1,pattern color=.} \gpfillpath;
\draw[gp path] (4.759,0.616)--(4.759,2.293)--(4.963,2.293)--(4.963,0.616)--cycle;
\def\gpfillpath{(5.782,0.616)--(5.988,0.616)--(5.988,2.302)--(5.782,2.302)--cycle}
\gpfill{color=gpbgfillcolor} \gpfillpath;
\gpfill{color=gp lt color 1,gp pattern 1,pattern color=.} \gpfillpath;
\draw[gp path] (5.782,0.616)--(5.782,2.301)--(5.987,2.301)--(5.987,0.616)--cycle;
\def\gpfillpath{(6.806,0.616)--(7.011,0.616)--(7.011,3.216)--(6.806,3.216)--cycle}
\gpfill{color=gpbgfillcolor} \gpfillpath;
\gpfill{color=gp lt color 1,gp pattern 1,pattern color=.} \gpfillpath;
\draw[gp path] (6.806,0.616)--(6.806,3.215)--(7.010,3.215)--(7.010,0.616)--cycle;
\gpcolor{color=gp lt color border}
\node[gp node right] at (6.780,4.238) {MLlib};
\def\gpfillpath{(6.964,4.161)--(7.880,4.161)--(7.880,4.315)--(6.964,4.315)--cycle}
\gpfill{color=gpbgfillcolor} \gpfillpath;
\gpfill{color=gp lt color 2,gp pattern 2,pattern color=.} \gpfillpath;
\gpcolor{color=gp lt color 2}
\gpsetlinetype{gp lt plot 2}
\draw[gp path] (6.964,4.161)--(7.880,4.161)--(7.880,4.315)--(6.964,4.315)--cycle;
\def\gpfillpath{(2.916,0.616)--(3.122,0.616)--(3.122,2.701)--(2.916,2.701)--cycle}
\gpfill{color=gpbgfillcolor} \gpfillpath;
\gpfill{color=gp lt color 2,gp pattern 2,pattern color=.} \gpfillpath;
\draw[gp path] (2.916,0.616)--(2.916,2.700)--(3.121,2.700)--(3.121,0.616)--cycle;
\def\gpfillpath{(3.940,0.616)--(4.145,0.616)--(4.145,3.006)--(3.940,3.006)--cycle}
\gpfill{color=gpbgfillcolor} \gpfillpath;
\gpfill{color=gp lt color 2,gp pattern 2,pattern color=.} \gpfillpath;
\draw[gp path] (3.940,0.616)--(3.940,3.005)--(4.144,3.005)--(4.144,0.616)--cycle;
\def\gpfillpath{(4.963,0.616)--(5.169,0.616)--(5.169,2.739)--(4.963,2.739)--cycle}
\gpfill{color=gpbgfillcolor} \gpfillpath;
\gpfill{color=gp lt color 2,gp pattern 2,pattern color=.} \gpfillpath;
\draw[gp path] (4.963,0.616)--(4.963,2.738)--(5.168,2.738)--(5.168,0.616)--cycle;
\def\gpfillpath{(5.987,0.616)--(6.192,0.616)--(6.192,2.875)--(5.987,2.875)--cycle}
\gpfill{color=gpbgfillcolor} \gpfillpath;
\gpfill{color=gp lt color 2,gp pattern 2,pattern color=.} \gpfillpath;
\draw[gp path] (5.987,0.616)--(5.987,2.874)--(6.191,2.874)--(6.191,0.616)--cycle;
\def\gpfillpath{(7.010,0.616)--(7.216,0.616)--(7.216,3.576)--(7.010,3.576)--cycle}
\gpfill{color=gpbgfillcolor} \gpfillpath;
\gpfill{color=gp lt color 2,gp pattern 2,pattern color=.} \gpfillpath;
\draw[gp path] (7.010,0.616)--(7.010,3.575)--(7.215,3.575)--(7.215,0.616)--cycle;
\gpcolor{color=gp lt color border}
\gpsetlinetype{gp lt border}
\draw[gp path] (1.688,3.587)--(1.688,0.616)--(7.829,0.616)--(7.829,3.587)--cycle;
%% coordinates of the plot area
\gpdefrectangularnode{gp plot 1}{\pgfpoint{1.688cm}{0.616cm}}{\pgfpoint{7.829cm}{3.587cm}}
\end{tikzpicture}
%% gnuplot variables

		\caption{The performance of FlashMatrix both in memory and on SSDs
		compared with Spark.}
		\label{perf:fm}
	\end{center}
\end{figure}

\subsection{Performance of FlashMatrix vs. R}

\subsection{Performance of FlashMatrix vs. C/C++}

\subsection{Effectiveness of optimizations}

In this section, we measure the effectiveness of optimizations in FlashMatrix,
i.e., matrix operation fusion to reduce data movement between CPU and SSDs and
using vectorized user-defined functions (VUDF) to reduce the overhead of
function calls. For operation fusion, we illustrate its effectiveness in both
main memory and CPU cache. We use GMM to illustrate the effectiveness of matrix
operation fusion because this application allows aggressive operation fusion.

Matrix operation fusion has very positive impact on both in-memory and
external-memory performance of GMM (Figure \ref{perf:mem_move}). Even though
GMM has the highest computation overhead among all applications in Section
\ref{sec:apps}, materializing every matrix operation on SSDs causes significant
performance degradation and SSDs are the main bottleneck. Fusing matrix
operations in memory significantly reduces the burden on SSDs and improves
performance by a factor of 5. Operation fusion in main memory also has noticeable
performance improvement when GMM runs in memory, although the impact is much
lower, owing to reduced memory allocation overhead. Operation fusion in
the CPU cache also has very positive performance improvement on in-memory
and external-memory execution of GMM by 30\% and 28\%, respectively, on top of
the improvement from fusion in main memory.

\begin{figure}
	\begin{center}
		\footnotesize
		\begin{tikzpicture}[gnuplot]
%% generated with GNUPLOT 4.6p4 (Lua 5.1; terminal rev. 99, script rev. 100)
%% Wed 30 Mar 2016 11:10:39 AM EDT
\path (0.000,0.000) rectangle (8.382,4.572);
\gpcolor{color=gp lt color border}
\gpsetlinetype{gp lt border}
\gpsetlinewidth{1.00}
\draw[gp path] (1.504,0.616)--(1.684,0.616);
\draw[gp path] (7.829,0.616)--(7.649,0.616);
\node[gp node right] at (1.320,0.616) { 0};
\draw[gp path] (1.504,1.149)--(1.684,1.149);
\draw[gp path] (7.829,1.149)--(7.649,1.149);
\node[gp node right] at (1.320,1.149) { 50};
\draw[gp path] (1.504,1.681)--(1.684,1.681);
\draw[gp path] (7.829,1.681)--(7.649,1.681);
\node[gp node right] at (1.320,1.681) { 100};
\draw[gp path] (1.504,2.214)--(1.684,2.214);
\draw[gp path] (7.829,2.214)--(7.649,2.214);
\node[gp node right] at (1.320,2.214) { 150};
\draw[gp path] (1.504,2.746)--(1.684,2.746);
\draw[gp path] (7.829,2.746)--(7.649,2.746);
\node[gp node right] at (1.320,2.746) { 200};
\draw[gp path] (1.504,3.279)--(1.684,3.279);
\draw[gp path] (7.829,3.279)--(7.649,3.279);
\node[gp node right] at (1.320,3.279) { 250};
\draw[gp path] (3.612,0.616)--(3.612,0.796);
\draw[gp path] (3.612,3.279)--(3.612,3.099);
\node[gp node center] at (3.612,0.308) {IM};
\draw[gp path] (5.721,0.616)--(5.721,0.796);
\draw[gp path] (5.721,3.279)--(5.721,3.099);
\node[gp node center] at (5.721,0.308) {EM};
\draw[gp path] (1.504,3.279)--(1.504,0.616)--(7.829,0.616)--(7.829,3.279)--cycle;
\node[gp node center,rotate=-270] at (0.246,1.947) {Runtime (s)};
\node[gp node right] at (3.382,4.238) {Base};
\def\gpfillpath{(3.566,4.161)--(4.482,4.161)--(4.482,4.315)--(3.566,4.315)--cycle}
\gpfill{color=gpbgfillcolor} \gpfillpath;
\gpfill{color=gp lt color 0,gp pattern 0,pattern color=.} \gpfillpath;
\gpcolor{color=gp lt color 0}
\gpsetlinetype{gp lt plot 0}
\draw[gp path] (3.566,4.161)--(4.482,4.161)--(4.482,4.315)--(3.566,4.315)--cycle;
\def\gpfillpath{(3.191,0.616)--(3.613,0.616)--(3.613,1.079)--(3.191,1.079)--cycle}
\gpfill{color=gpbgfillcolor} \gpfillpath;
\gpfill{color=gp lt color 0,gp pattern 0,pattern color=.} \gpfillpath;
\draw[gp path] (3.191,0.616)--(3.191,1.078)--(3.612,1.078)--(3.612,0.616)--cycle;
\def\gpfillpath{(5.299,0.616)--(5.722,0.616)--(5.722,3.026)--(5.299,3.026)--cycle}
\gpfill{color=gpbgfillcolor} \gpfillpath;
\gpfill{color=gp lt color 0,gp pattern 0,pattern color=.} \gpfillpath;
\draw[gp path] (5.299,0.616)--(5.299,3.025)--(5.721,3.025)--(5.721,0.616)--cycle;
\gpcolor{color=gp lt color border}
\node[gp node right] at (3.382,3.930) {Mem-fuse};
\def\gpfillpath{(3.566,3.853)--(4.482,3.853)--(4.482,4.007)--(3.566,4.007)--cycle}
\gpfill{color=gpbgfillcolor} \gpfillpath;
\gpfill{color=gp lt color 1,gp pattern 1,pattern color=.} \gpfillpath;
\gpcolor{color=gp lt color 1}
\gpsetlinetype{gp lt plot 1}
\draw[gp path] (3.566,3.853)--(4.482,3.853)--(4.482,4.007)--(3.566,4.007)--cycle;
\def\gpfillpath{(3.612,0.616)--(4.035,0.616)--(4.035,1.011)--(3.612,1.011)--cycle}
\gpfill{color=gpbgfillcolor} \gpfillpath;
\gpfill{color=gp lt color 1,gp pattern 1,pattern color=.} \gpfillpath;
\draw[gp path] (3.612,0.616)--(3.612,1.010)--(4.034,1.010)--(4.034,0.616)--cycle;
\def\gpfillpath{(5.721,0.616)--(6.143,0.616)--(6.143,1.058)--(5.721,1.058)--cycle}
\gpfill{color=gpbgfillcolor} \gpfillpath;
\gpfill{color=gp lt color 1,gp pattern 1,pattern color=.} \gpfillpath;
\draw[gp path] (5.721,0.616)--(5.721,1.057)--(6.142,1.057)--(6.142,0.616)--cycle;
\gpcolor{color=gp lt color border}
\node[gp node right] at (6.506,4.238) {Cache-fuse};
\def\gpfillpath{(6.690,4.161)--(7.606,4.161)--(7.606,4.315)--(6.690,4.315)--cycle}
\gpfill{color=gpbgfillcolor} \gpfillpath;
\gpfill{color=gp lt color 2,gp pattern 2,pattern color=.} \gpfillpath;
\gpcolor{color=gp lt color 2}
\gpsetlinetype{gp lt plot 2}
\draw[gp path] (6.690,4.161)--(7.606,4.161)--(7.606,4.315)--(6.690,4.315)--cycle;
\def\gpfillpath{(4.034,0.616)--(4.457,0.616)--(4.457,0.891)--(4.034,0.891)--cycle}
\gpfill{color=gpbgfillcolor} \gpfillpath;
\gpfill{color=gp lt color 2,gp pattern 2,pattern color=.} \gpfillpath;
\draw[gp path] (4.034,0.616)--(4.034,0.890)--(4.456,0.890)--(4.456,0.616)--cycle;
\def\gpfillpath{(6.142,0.616)--(6.565,0.616)--(6.565,0.937)--(6.142,0.937)--cycle}
\gpfill{color=gpbgfillcolor} \gpfillpath;
\gpfill{color=gp lt color 2,gp pattern 2,pattern color=.} \gpfillpath;
\draw[gp path] (6.142,0.616)--(6.142,0.936)--(6.564,0.936)--(6.564,0.616)--cycle;
\gpcolor{color=gp lt color border}
\gpsetlinetype{gp lt border}
\draw[gp path] (1.504,3.279)--(1.504,0.616)--(7.829,0.616)--(7.829,3.279)--cycle;
%% coordinates of the plot area
\gpdefrectangularnode{gp plot 1}{\pgfpoint{1.504cm}{0.616cm}}{\pgfpoint{7.829cm}{3.279cm}}
\end{tikzpicture}
%% gnuplot variables

		\caption{The effectiveness of matrix operation fusion in main memory
			(mem-fuse) and in CPU cache (cache-fuse). The operation fusion in
		CPU cache is applied on top of the fusion in main memory.}
		\label{perf:mem_move}
	\end{center}
\end{figure}

Overhead of function calls.
